\documentclass[]{article}
\usepackage{lmodern}
\usepackage{amssymb,amsmath}
\usepackage{ifxetex,ifluatex}
\usepackage{fixltx2e} % provides \textsubscript
\ifnum 0\ifxetex 1\fi\ifluatex 1\fi=0 % if pdftex
  \usepackage[T1]{fontenc}
  \usepackage[utf8]{inputenc}
\else % if luatex or xelatex
  \ifxetex
    \usepackage{mathspec}
  \else
    \usepackage{fontspec}
  \fi
  \defaultfontfeatures{Ligatures=TeX,Scale=MatchLowercase}
\fi
% use upquote if available, for straight quotes in verbatim environments
\IfFileExists{upquote.sty}{\usepackage{upquote}}{}
% use microtype if available
\IfFileExists{microtype.sty}{%
\usepackage{microtype}
\UseMicrotypeSet[protrusion]{basicmath} % disable protrusion for tt fonts
}{}
\usepackage[margin=1in]{geometry}
\usepackage{hyperref}
\hypersetup{unicode=true,
            pdfborder={0 0 0},
            breaklinks=true}
\urlstyle{same}  % don't use monospace font for urls
\usepackage{graphicx,grffile}
\makeatletter
\def\maxwidth{\ifdim\Gin@nat@width>\linewidth\linewidth\else\Gin@nat@width\fi}
\def\maxheight{\ifdim\Gin@nat@height>\textheight\textheight\else\Gin@nat@height\fi}
\makeatother
% Scale images if necessary, so that they will not overflow the page
% margins by default, and it is still possible to overwrite the defaults
% using explicit options in \includegraphics[width, height, ...]{}
\setkeys{Gin}{width=\maxwidth,height=\maxheight,keepaspectratio}
\IfFileExists{parskip.sty}{%
\usepackage{parskip}
}{% else
\setlength{\parindent}{0pt}
\setlength{\parskip}{6pt plus 2pt minus 1pt}
}
\setlength{\emergencystretch}{3em}  % prevent overfull lines
\providecommand{\tightlist}{%
  \setlength{\itemsep}{0pt}\setlength{\parskip}{0pt}}
\setcounter{secnumdepth}{0}
% Redefines (sub)paragraphs to behave more like sections
\ifx\paragraph\undefined\else
\let\oldparagraph\paragraph
\renewcommand{\paragraph}[1]{\oldparagraph{#1}\mbox{}}
\fi
\ifx\subparagraph\undefined\else
\let\oldsubparagraph\subparagraph
\renewcommand{\subparagraph}[1]{\oldsubparagraph{#1}\mbox{}}
\fi

%%% Use protect on footnotes to avoid problems with footnotes in titles
\let\rmarkdownfootnote\footnote%
\def\footnote{\protect\rmarkdownfootnote}

%%% Change title format to be more compact
\usepackage{titling}

% Create subtitle command for use in maketitle
\newcommand{\subtitle}[1]{
  \posttitle{
    \begin{center}\large#1\end{center}
    }
}

\setlength{\droptitle}{-2em}
  \title{}
  \pretitle{\vspace{\droptitle}}
  \posttitle{}
  \author{}
  \preauthor{}\postauthor{}
  \date{}
  \predate{}\postdate{}


\begin{document}

+++ image = ``img/portfolio/peace.png'' showonlyimage = false date =
``2016-11-05T19:44:32+05:30'' title = ``Electrode position Trial'' draft
= false weight = 3 +++

``The Electrode position trial: Comparison of Anterior-posterior versus
anterior-lateral patch position for cardioversion''

Referat fra mødet den 29.06.2018.

Til stede: Bo Løfgren (BL), Kasper Glerup (KG), Andi Albertsen (AA),
Anders Schmidt (AS).

\textbf{Studie-design og protokol}

AD Hypotese/forskningsspørgsmål: Er Anterior-posterior eletrodeplacering
superiort til anterior-lateral elektrodeplacering ved kardiovertering af
atrieflimren? AD Design: Vi aftaler at gå videre med et to-arms
randomiseret studie med forskellige elektrodeplacering, men samme
energiprotokol, der er eskalerende, startende på 100 J og som går til
360 J i begge grupper. AD Sample size: Kan med fordel beregnes på
``first shock'' ved lav energi (100J), hvorved succes raten vil være
lav. Dette giver mulighed for at afvise en nulhypotese med en forskel på
\textgreater{}10\% med en fornuftigt antal patienter (ca. 300). AS
udarbejder et udkast til en protokol (deadline 18.07.2018). Følger
desuden op med udkast til Videnskabsetisk godkendelse,
patientinformation, samtykke, database-udkast/RedCap
(variable-oversigt), randomiseringsmodul (elektronisk), udkast til
registrering på Clinical Trials

\textbf{Formalia, akademisk kreditering}

Vi aftaler at projektet har AA som sidste forfatter (Viborg initieret
idé), AS er ``primus motor'' og 1. forfatter, KG er 2. forfatter, da han
vil bidrage betydeligt i forhold til studiet drift i øvrigt, BL er 2.
sidste forfatter/medforfatter. Øvrige lokale studie-aktive læger vil
kunne komme på i forfatterblokken senere.

\textbf{Økonomi}

AA har allerede skaffet kr. 125.000 til projektet. AS udarbejder et
udkast til budget, sender en fundsoversigt til AA og AA vil gå videre
med at søge funding. Budgettet har vi aftalt skal indeholde posterne:
VIP-løn, frikøb af senior studieansvarlig på de enkelte rekruterende
enheder (dvs. 3 stk; tiltænkt Andi, Bo, og ``Mr.~X'') TAP-løn,
projektsygeplejerske i x antal timer

\textbf{Nyt investigator-møde} Vi aftaler at følge op med et nyt møde
efter sommeren, fx. i Aarhus. AA og AS vil dog desuden mødes ``ad hoc''.


\end{document}
